\documentclass[12 pt, letterpaper]{article}
% \usepackage[margin = 1 in]{geometry}
\usepackage[square, numbers]{natbib}
\usepackage{amsfonts}
\usepackage{amsmath}
\usepackage{mathtools}
\usepackage{xspace}




\newcommand{\bigO}{\ensuremath{\mathcal{O}\xspace}}


\author{James McFeeters}
\title{Colloquium Project Proposal: Kleene's Theorem}
\date{February 5, 2018}
\begin{document}
\maketitle

\setlength{\parindent}{0 pt}
\setlength{\parskip}{1 em}

I will present a proof of Kleene's Theorem, an important result in formal computer science that establishes the equivalence of finite automata and regular languages.
This will require presenting an introduction to formal language theory and the proofs of some related results, such as the equivalence of deterministic and nondeterministic finite automata. 
My paper and presentation will include original examples of the conversion of finite automata to regular expressions, and vice-versa, as well as examples demonstrating some of the basic concepts of formal language theory.

I have selected only two sources at this time. 
I believe that they will be enough to inform the bulk of my project, but I will add more if the need arises. 
Each includes a proof of Kleene's theorem, as well as the proofs of the related results necessary for my project.
The majority of the work involved in my project will be the creation of original examples, with which more sources will not help.

\clearpage
\nocite{*}
\bibliographystyle{plainnat}
\bibliography{references.bib}
\end{document}
