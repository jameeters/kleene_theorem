\documentclass{bcthesis}

%%% Other packages needed by your document may be loaded here.
\usepackage{url}              % For formatting URLs 
\usepackage{booktabs}         % Publication-quality tables.
\usepackage{natbib}           % Provides some nice citation and
							  % bibliography formatting commands.
\bibpunct[:~]{(}{)}{;}{a}{,}{,~} % Set some defaults for bibliographic
								 % punctuation used by natbib.sty.
\usepackage{verbatim}
\usepackage{graphicx}
\usepackage{calc}
\usepackage{subfig}
\usepackage{textcomp}
\usepackage{amsfonts}
\usepackage{tikz}
\usepackage{standalone}
\usepackage{showframe}
\usepackage{james}
\usepackage[plainpages=false,pdfpagelabels, hidelinks]{hyperref}

\usetikzlibrary{arrows, automata}

%%% Document Settings
	% The following command allows tables to split between pages:
	\allowdisplaybreaks

	% Provide additional context around errors. 
	\setcounter{errorcontextlines}{1000}

	\setlength{\parskip}{0.5 em}

	\newif\ifbuildfrontmatter
	\buildfrontmatterfalse


%%% End document settings

%%% Begin information section 
	%% title of your report?
	\title{A Presentation of Kleene's Theorem with Examples}

	%% Your name 
	\author{James McFeeters}

	% Photo -- see bcthesis class line 168

	\advisor{Darrah Chavey}

	%% Second reader's name? Ask the Colloquium Instructor who it will be.
	\reader{Name of Second Reader}

	\thesisyear{2018}
	\thesismonth{May}
%%% End of information section.


%%% New Environments

	% Environment for finite automata construction examples
	% Places explanatory text next to the relevant diagram
	\newenvironment{exstep}[1]
		{
			\begin{minipage}{0.5 \textwidth}
				#1
			\end{minipage} \begin{minipage}{0.5 \paperwidth}
		}
		{
			\end{minipage}\\[0.5 em]
		}


%%% End new environments 



%%% New commands

%%% End new commands


%%% The start of the document!
\begin{document}

\ifbuildfrontmatter
\frontmatter

	\maketitle

	\begin{abstract}
		Your abstract goes here.
	\end{abstract}


	%%% If you want to thank someone for their influence on your life or your work, here's where you'd do it.
	\begin{acknowledgments}
		Here you should acknowledge any special help on your project, or particular influences on your life or mathematical development.
	\end{acknowledgments}

	%%% Table of Contents, List of Figures, and List of Tables. 
	% If you don't have any figures or tables in your report, you can
	% comment out the appropriate command.
	\tableofcontents
	% \listoffigures
	% \listoftables
%%% End of the front matter.
\fi

%%% Beginnning of the main matter.
\mainmatter

\chapter{Introduction}%
\label{ch:introduction}

	\section{Introduction to Formal Language} % (fold)
	\label{sec:introduction_to_formal_language}

		Formal language theory is the study of sets of strings comprised of some 

	% section introduction_to_formal_language (end)

	\section{Regular Expressions} % (fold)
	\label{sec:regular_expressions}
		
	% section regular_expressions (end)

	\section{Finite Automata} % (fold)
	\label{sec:finite_automata}
		A finite automaton is constructed as a 5-tuple
		\[
			M = (Q, q_0, \Sigma, \delta, F)
		\]
		$Q$ is the set of states in the machine.
		$q_0$ is the initial state of the machine, this is the state of the machine before any input symbol is read.
		$\Sigma$ is the alphabet of possible input symbols to the machine.
		$\delta: Q \times \Sigma \mapsto Q$ is the transition function.
		Given a current state and an input symbol from $\Sigma$, $\delta$ outputs the next state of the machine.
		$F$ is the set of accepting states.
		If, when the last input symbol has been read, the machine's state is an accepting state, the machine is said to accept that input.
	% section finite_automata (end)

	\section{Finite Automata Construction Example} % (fold)
	\label{sec:finite_automata_construction_example}
		We will construct a simple finite automaton to determine if a binary string represents a number divisible by three.
		Denote this $M_3 = (Q, q_0, \Sigma, \delta, F)$.
		Since the inputs are to be binary strings, we know $\Sigma = \{ 0, 1 \}$.
		We can also assume that each state of $M_3$ should represent the value of the current input modulo 3, so $Q = \{q_0, q_1, q_2\}$.
		Now we need an initial state. 
		We'll assume that the machine starts at $q_0$. 
		This makes sense, because we could achieve the same result by claiming to prefix every input string with a 0, as this would not change its value.
		Obviously, all numbers divisible by 3 will have a value of 0 modulo 3, so $F = \{q_0\}$.

		Now the only part of $M_3$ left undefined is $\delta$, the set of state transitions.

			\begin{exstep}
				{
					We begin with only the initial state.
					The input so far is $\lambda$.
				}
				\includestandalone{tikz/mod3_construction/step_1}
			\end{exstep}

			\begin{exstep}
				{
					With the 0 as the current state, suppose the first input symbol is 0. 
					Then the number so far is $0_b = 0 = 0 \mod 3$, so we add the following to $\delta$: $(q_0, 0, q_0)$.
				}
				\includestandalone{tikz/mod3_construction/step_2}
			\end{exstep}

			\begin{exstep}
				{
					Now suppose that the first input symbol is 1. 
					Then $1_b = 1 \mod 3$, so we need to add state $q_1$ to our drawing, and add $(q_0, 1, q_1)$ to $\delta$.
				}
				\includestandalone{tikz/mod3_construction/step_3}
			\end{exstep}

			\begin{exstep}
				{
					Now we have all the transitions we need from $q_0$, so we move on to $q_1$.
					We'll assume that the input so far is $1_b$.
					Then if the next input symbol is $0$, we have $10_b = 2 = 2 \mod 3$.
					So we add state $q_2$ to the diagram, and add $(q_1, 0, q_2)$ to $\delta$.
				}
				\includestandalone{tikz/mod3_construction/step_4}
			\end{exstep}

			\begin{exstep}
				{
					But if the next input is $1$, then we have $11_b = 3 = 0 \mod 3$.
					So we add $(q_1, 1, q_0)$ to $\delta$.
				}
				\includestandalone{tikz/mod3_construction/step_5}
			\end{exstep}
			

			\begin{exstep}
				{
				Now we have all possible transitions from $q_1$, so we move on to $q_2$.
				Assume that the input so far is $10_b$.
				if the next symbol is $0$, we will have $100_b = 1 \mod 3$, so we add $(q_2, 0, q_1)$ to $\delta$.
				}
				\includestandalone{tikz/mod3_construction/step_6}
			\end{exstep}
			
			\begin{exstep}
				{
					But if the next input is $1$, we will have $101_b = 5 = 2 \mod 3$.
					So we add $(q_2, 1, q_2)$ to $\delta$.
				}
				% import last tikz drawing here
				\includestandalone{tikz/mod3_construction/step_7}
			\end{exstep}

	% section finite_automata_construction_example (end)

\chapter{Main Body}%
\label{sec:structured-writing}

Again, section as needed.

\chapter{Mathematical Notation}%
\label{sec:mathematical-notation}

You may find it useful to the reader to provide a table of notation. 

\chapter{Conclusion(s)}

%\appendix
%%% Any appendices are delineated with the \appendix command.
%%% Individual appendices are begun with the standard \chapter or
%%% \section commands.  Use an appendix for any very long argument or side issue.

\backmatter

\nocite{*}
\bibliographystyle{plainnat}
\bibliography{references.bib}

\end{document}


