\documentclass[12 pt, twoside, letterpaper]{article}

\usepackage[margin = 1 in]{geometry}
\usepackage{amsthm}
\usepackage{amsmath}
\usepackage{amsfonts}
\usepackage{enumitem}
\usepackage[usenames]{xcolor}


\newcommand{\todo}[1]{{\textbf{todo: \color{red} {#1}}}}
\newcommand{\note}[1]{\textit{\color{blue}{#1}}}
\newcommand{\meo}[1]{#1}
% \renewcommand{\meo}[1]{\iffalse #1 \fi}

\newcommand{\refthm}[1]{Theorem~\ref{#1}}
\newcommand{\refprop}[1]{Claim~\ref{#1}}

\renewcommand{\qedsymbol}{}


%% Ams Theorem setup
\theoremstyle{definition}
\newtheorem{definition}{Definition}

\theoremstyle{remark}
\newtheorem*{remark}{Remark}

\theoremstyle{plain}
\newtheorem*{example}{Example}
\newtheorem{theorem}{Theorem}
\newtheorem{claim}{Claim}
% \newtheorem*{kleenethm}{Kleene's Theorem}

\title{Kleene's Theorem: Outline}
\author{James McFeeters}
\date{\today}
\begin{document}
\maketitle

\section{Introduction} % (fold)
\label{sec:introduction}

	\subsection{Basic Definitions} % (fold)
	\label{sub:basic_definitions}

		\begin{definition}[Symbol]
			
		\end{definition}

		\begin{definition}[Alphabet]

		\end{definition}

		\begin{definition}[String]
			Define string, length of a string, and the empty string.
		\end{definition}

		\begin{definition}[Language]
			A \textit{language} over some alphabet $\Sigma$ is any set of strings over $\Sigma$.
		\end{definition}

		\begin{remark}
			The empty set and $\{ \lambda \}$ as languages.
		\end{remark}
	% subsection basic_definitions (end)

	\subsection{Regular Expressions} % (fold)
	\label{sub:regular_expressions}
		
		\begin{definition}[Concatenation]
			Define concatenation of symbols and words and give the corresponding notation.
			\meo{
				Make sure to include the notation $x^i$ and $x^0$.
			}
		\end{definition}

		\begin{definition}[Concatenation of Sets]
			Give the definition of the concatenation of sets (set product). 
		\end{definition}

		\begin{remark}
			Note that concatenation is an associative binary operation, and that the set of strings over an alphabet is closed under concatenation, as is the set of languages.
			\meo{
				Maybe also note that length has the properties of the logarithm with regard to concatenation.
				It's a curiosity, but not really useful information
			}
		\end{remark}

		\begin{definition}[Positive Closure]
			Define the positive closure of a set and give notation. 
			Also give notation and meaning for the set of nonempty words over an alphabet.
			\note{I'm not sure I actually need to include this.}
		\end{definition}

		\begin{definition}[Kleene Closure]
			Define the Kleene closure of a symbol, string, and set and give notation.
			Note that $\Sigma^*$ is the set of all words over an alphabet.
		\end{definition}

		\meo{
			Add a definition of regular operations. 
			Say that the regular operations are concatenation and Kleene closure plus union.

			Then define regular languages as in the longhand, and follow it up with the definition of regular expressions as a shorthand for regular expressions (also as in the longhand).
		}

		\begin{definition}[Atomic Languages]
			Define the atomic languages $\emptyset$, $\{ \lambda \}$, and $\{ a \}, \ \forall a \in \Sigma$.
			This is the basis for the recursive definition of regular expressions
		\end{definition}

		\begin{definition}[Regular Expressions]
			State that the regular operations are 
			\begin{enumerate}[label=(\roman*), itemsep = -0.3 ex]
				\item Concatenation
				\item Union
				\item Kleene Closure
			\end{enumerate}
		\end{definition}

		\begin{definition}[Regular Language]
			A regular language is a set of strings obtained from the application of finitely many regular operations to atomic languages.
		\end{definition}
	% subsection regular_expressions (end)

	\subsection{Finite Automata} % (fold)
	\label{sub:finite_automata}
	
		\begin{definition}[Finite Automaton]
			Give the 5-tuple definition of a finite automaton, and explain each of the components.
		\end{definition}

		\begin{remark}
			Mention that the transition function is defined on strings as well as symbols, and show what that means.
		\end{remark}

		\begin{definition}(Acceptance by Finite Automaton)
			Give a definition and notation of what it means for a FA to accept an input string.
			Also give notation for the language of all strings accepted by a FA.
			The language of all strings accepted by a FA is said to be represented by that FA.
			A language that can be represented by a FA is called \textit{representable}.
		\end{definition}

		\begin{example}[Finite Automata Construction]
			Give an illustrated step-by-step explanation of the construction of a finite automaton to recognize binary numbers divisible by three.
		\end{example}

		\begin{definition}[Deterministic and Nondeterministic Finite Automata]
			Explain the distinction between deterministic and nondeterministic finite automata.
		\end{definition}

	% subsection finite_automata (end)

% section introduction (end)

\section{Kleene's Theorem} % (fold)
\label{sec:kleenes_theorem}

	\begin{theorem}[Kleene's Theorem]
		Give a statement of Kleene's Theorem to motivate the section.
	\end{theorem}

	\begin{claim}
	\label{prop:regular_languages_representable}
		All regular languages are representable.
	\end{claim}

	\begin{proof}
		Proof of \refprop{prop:regular_languages_representable}
	\end{proof}

	\begin{claim}
	\label{prop:ndfa_dfa_equivalent}
		A language is representable by a NDFA if and only if it is representable by a DFA.
	\end{claim}

	\begin{proof}
	\label{proof:ndfa_dfa_equivalent}
		Proof of \refprop{prop:ndfa_dfa_equivalent}.
	\end{proof}

	\begin{claim}
	\label{prop:representable_languages_regular}
		All representable languages are regular.
	\end{claim}

	\begin{proof}
		Proof of \refprop{prop:representable_languages_regular}
	\end{proof}
	

	This constitutes a proof of Kleene's theorem.

% section kleenes_theorem (end)


\section{Examples Illustrating Kleene's Theorem} % (fold)
\label{sec:examples}

	\begin{example}[Mod 3 Finite Automaton]
		Constructing regular expression from the mod 3 finite automaton.
		This will help explain the proof that all representable languages are regular.
	\end{example}

	\meo{
		Figure out if there are any other parts of the proof that deserve an example to help explain them.
		I will probably include pictures for regular language $\iff$ representable.
	}

% section examples (end)
\meo{

	Look into any background or motivation that might be good to include.
	Anything about the importance of Kleene's theorem to the implementation of actual software might be good to include.
	But I worry about including too much practical CS in what really should be a math paper.
}



\end{document}
