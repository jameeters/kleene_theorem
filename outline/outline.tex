\documentclass[12 pt, twoside, letterpaper]{article}

\usepackage[margin = 1 in]{geometry}
\usepackage{amsthm}
\usepackage{amsmath}
\usepackage{amsfonts}
\usepackage[usenames]{xcolor}

\newcommand{\todo}[1]{{\textbf{todo: \color{red} {#1}}}}

\newcommand{\refthm}[1]{Theorem~\ref{#1}}


%% Ams Theorem setup
\theoremstyle{definition}
\newtheorem{definition}{Definition}

\theoremstyle{plain}
\newtheorem*{example}{Example}
\newtheorem{theorem}{Theorem}
% \newtheorem*{kleenethm}{Kleene's Theorem}



\title{Kleene's Theorem: Outline}
\author{James McFeeters}
\date{\today}
\begin{document}
\maketitle

\section{Introduction} % (fold)
\label{sec:introduction}

	\subsection{Basic Definitions} % (fold)
	\label{sub:basic_definitions}

		\begin{definition}[Symbol]
			
		\end{definition}

		\begin{definition}[Alphabet]

		\end{definition}

		\begin{definition}[String]
			Define string, length of a string, and the empty string.
		\end{definition}

		\begin{definition}[Language]
			Give a minimal definition of a language.
		\end{definition}
	% subsection basic_definitions (end)

	\subsection{Regular Expressions} % (fold)
	\label{sub:regular_expressions}
		
		\begin{definition}[Concatenation]
			Define concatenation of strings and sets (set product) and give corresponding notation.
		\end{definition}

		\begin{definition}[Positive Closure]
			Define the positive closure of a set and give notation. 
			Also give notation and meaning for the set of nonempty words over an alphabet.
		\end{definition}

		\begin{definition}[Kleene Closure]
			Define the Kleene closure of a set and give notation.
			Also give notation and meaning for the set of empty and nonempty words over an alphabet.
		\end{definition}

		\begin{definition}[Recursive Definition]
			Give the recursive definition of regular expressions over an alphabet.
		\end{definition}
	% subsection regular_expressions (end)

	\subsection{Finite Automata} % (fold)
	\label{sub:finite_automata}
	
		\begin{definition}[Finite Automaton]
			Give the 5-tuple definition of a finite automaton, and explain each of the components.
		\end{definition}

		\begin{definition}[Path]
			\todo{Change the name of this definition.}
			Give a definition and notation for the sequence of states corresponding to some input string.
		\end{definition}

		\begin{definition}[Acceptance]
			\todo{Change the name of this definition.}
			Give a definition and notation of what it means for a FA to accept an input string.
			Also give notation for the language of all strings accepted by a FA.
		\end{definition}

		\begin{example}[Finite Automata Construction]
			Give an illustrated step-by-step explanation of the construction of a finite automaton to recognize binary numbers divisible by three.
		\end{example}

	% subsection finite_automata (end)

% section introduction (end)

\section{Main Body} % (fold)
\label{sec:main_body}

	\begin{theorem}[Kleene's Theorem]
		Give a statement of Kleene's Theorem to motivate the section.
	\end{theorem}

	\begin{definition}[Deterministic and Nondeterministic Finite Automata]
		Explain the distinction between deterministic and nondeterministic finite automata.
	\end{definition}

	\begin{theorem}
	\label{thm:ndfa_dfa_equivalent}
		A language is representable by a NDFA if and only if it is representable by a DFA.
	\end{theorem}

	\begin{proof}
	\label{proof:ndfa_dfa_equivalent}
		Proof of \refthm{thm:ndfa_dfa_equivalent}.
	\end{proof}

	\begin{theorem}
	\label{thm:regular_languages_representable}
		All regular languages are representable.
	\end{theorem}

	\begin{proof}
		Proof of \refthm{thm:regular_languages_representable}
	\end{proof}

	\begin{theorem}
	\label{thm:representable_languages_regular}
		All representable languages are regular.
	\end{theorem}

	This constitutes a proof of Kleene's theorem.






% section main_body (end)

\end{document}
