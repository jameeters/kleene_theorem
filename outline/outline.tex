\documentclass[12 pt, twoside, letterpaper]{article}

\usepackage[margin = 1 in]{geometry}
\usepackage{amsthm}
\usepackage{amsmath}
\usepackage{amsfonts}
\usepackage{enumitem}
\usepackage[usenames]{xcolor}


\newcommand{\todo}[1]{{\textbf{todo: \color{red} {#1}}}}
\newcommand{\note}[1]{\textit{\color{blue}{#1}}}
\newcommand{\meo}[1]{\texttt{#1}}
\renewcommand{\meo}[1]{\iffalse #1 \fi}
\newcommand{\bigo}{\mathcal{O}}

\newcommand{\refthm}[1]{Theorem~\ref{#1}}
\newcommand{\refprop}[1]{Claim~\ref{#1}}

\renewcommand{\qedsymbol}{}


%% Ams Theorem setup
\theoremstyle{definition}
\newtheorem{definition}{Definition}

\theoremstyle{remark}
\newtheorem*{remark}{Remark}

\theoremstyle{plain}
\newtheorem*{example}{Example}
\newtheorem{theorem}{Theorem}
\newtheorem{claim}{Claim}
% \newtheorem*{kleenethm}{Kleene's Theorem}

\title{Colloquium Paper Outline: \\[1 ex] Kleene's Theorem}
\author{James McFeeters}
\date{\today}
\begin{document}
\maketitle



\section{Basic Definitions} % (fold)
\label{sec:basic_definitions}

	\begin{definition}[Symbol]
		
	\end{definition}

	\begin{definition}[Alphabet]

	\end{definition}

	\begin{definition}[String]
		Define string, length of a string, and the empty string, and give the necessary notation.
	\end{definition}

	\begin{definition}[Language]
		A \textit{language} over some alphabet $\Sigma$ is any set of strings over $\Sigma$.
	\end{definition}

	\begin{remark}
		The empty set and $\{ \lambda \}$ as languages.
	\end{remark}
% section basic_definitions (end)

\section{Regular Expressions} % (fold)
\label{sec:regular_expressions}
	
	\begin{definition}[Concatenation]
		Define concatenation of symbols and words and give the corresponding notation.
		\meo{
			Make sure to include the notation $x^i$ and $x^0$.
		}
	\end{definition}

	\begin{definition}[Concatenation of Languages]
		Give the definition of the concatenation (product) of two languages.
	\end{definition}

	\begin{remark}
		Note that concatenation is an associative binary operation, and that the set of strings over an alphabet is closed under concatenation, as is the set of languages.
		\meo{
			Maybe also note that length has the properties of the logarithm with regard to concatenation.
			It's a curiosity, but not really useful information
		}
	\end{remark}

	\iffalse
	\begin{definition}[Positive Closure]
		Define the positive closure of a set and give notation. 
		Also give notation and meaning for the set of nonempty words over an alphabet.
		\note{I'm not sure I actually need to include this.}
	\end{definition}
	\fi

	\begin{definition}[Kleene Closure]
		Define the Kleene closure of a language and give notation.
		Note how the definition of the Kleene closure applies to symbols and strings, and that $\Sigma^*$ is the set of all words over an alphabet.
	\end{definition}

	\begin{definition}[Regular Operations]
		State that the regular operations are 
		\begin{enumerate}[label=(\roman*), itemsep = -0.3 ex]
			\item Concatenation
			\item Union
			\item Kleene Closure
		\end{enumerate}
	\end{definition}

	\begin{definition}[Regular Language]
		Given an alphabet $\Sigma$
		\begin{enumerate}[label=(\roman*), itemsep = -0.3 ex]
			\item $\emptyset$ is an atomic regular language over $\Sigma$.
			\item $\{ \lambda \}$ is an atomic regular language over $\Sigma$.
			\item $\{ a \}$ is an atomic regular language over $\Sigma$ for all $a \in \Sigma$.
		\end{enumerate}
		And any language obtained from an atomic regular language through a finite number of regular operations is a regular language over $\Sigma$.
	\end{definition}

	We now introduce \textit{regular expressions}, a shorthand for describing regular languages.


	\begin{definition}[Regular Expressions]
		The atomic regular expressions over any alphabet $\Sigma$ are defined as follows:
		\begin{enumerate}[label=(\roman*), itemsep = -0.3 ex]
			\item $\emptyset$ denotes the empty language.
			\item $\lambda$ denotes the language $\{ \lambda \}$.
			\item $a$ denotes the language $\{ a \}$ for all $a \in \Sigma$.
		\end{enumerate}

		If $\alpha$ and $\beta$ are regular expressions representing languages $A$ and $B$, then the following are also regular expressions:
		\begin{enumerate}[label=(\roman*), itemsep = -0.3 ex]
			\item $\alpha \beta$ denotes the language $AB$.
			\item $\alpha + \beta$ denotes the language $A \cup B$.
			\item $\alpha^*$ denotes the language $A^*$.
		\end{enumerate}
	\end{definition}

	\begin{remark}
		Note the distinction between regular languages and regular expressions: that one is a shorthand for writing the other.
	\end{remark}

	\meo{
		I could mention here that regular expressions are an example of a grammar, shockingly belonging to the class of regular grammars.
	}

% section regular_expressions (end)

\section{Finite Automata} % (fold)
\label{sec:finite_automata}

	\begin{definition}[Finite Automaton]
		Give the 5-tuple definition of a finite automaton, and explain each of the components.
		Mention the abbreviation ``FA.''
	\end{definition}

	\begin{remark}
		Mention that the transition function is defined on strings as well as symbols, and show what that means.
	\end{remark}

	\begin{definition}(Acceptance by Finite Automaton)
		Give a definition and notation of what it means for a FA to accept an input string.
		Also give notation for the language of all strings accepted by a FA.
		Define what it means for a language to be representable.
	\end{definition}

	\begin{example}[Finite Automata Construction]
		Give an illustrated step-by-step explanation of the construction of a finite automaton to recognize binary numbers divisible by three.
	\end{example}

	\begin{definition}[Deterministic and Nondeterministic Finite Automata]
		Explain the distinction between deterministic and nondeterministic finite automata.
		Mention the abbreviations ``DFA'' and ``NDFA.''
	\end{definition}

% section finite_automata (end)

\section{Kleene's Theorem} % (fold)
\label{sec:kleenes_theorem}

	\begin{theorem}[Kleene's Theorem]
		A language is representable if and only if it is regular.
	\end{theorem}

	\begin{claim}
	\label{prop:regular_languages_representable}
		All regular languages are representable.
	\end{claim}

	\begin{proof}
		Proof of \refprop{prop:regular_languages_representable}
	\end{proof}

	\begin{claim}
	\label{prop:ndfa_dfa_equivalent}
		A language is representable by a NDFA if and only if it is representable by a DFA.
	\end{claim}

	\begin{proof}
	\label{proof:ndfa_dfa_equivalent}
		Proof of \refprop{prop:ndfa_dfa_equivalent}.
	\end{proof}

	\begin{claim}
	\label{prop:representable_languages_regular}
		All representable languages are regular.
	\end{claim}

	\begin{proof}
		Proof of \refprop{prop:representable_languages_regular}
	\end{proof}
	

	This completes the proof of Kleene's theorem.

% section kleenes_theorem (end)


\section{Examples Illustrating Kleene's Theorem} % (fold)
\label{sec:examples}

	\begin{example}[Mod 3 Finite Automaton]
		Constructing regular expression from the mod 3 finite automaton.
		This will help explain the proof that all representable languages are regular.

		This example will also illustrate how the numbering of the states of an FA in the inductive step of the proof of \refprop{prop:representable_languages_regular} impacts the resulting regular expression (although all possible results are equivalent, they may look very different).
	\end{example}

	\meo{
		Figure out if there are any other parts of the proof that deserve an example to help explain them.
		I will probably include pictures for regular language $\iff$ representable in the initial proof.
	}

% section examples (end)

\section{Motivation} % (fold)
\label{sec:motivation}
	\begin{remark}
		We have seen that any regular expression is convertible into a NDFA and that any NDFA is convertible into a DFA.
		Although we have not mentioned it, there is also an algorithm to minimize any DFA: producing an equivalent automaton with a minimal number of states. 
		With a minimal DFA accepting a regular language, we can use a transition matrix to search through data for strings of that language in $\bigo(N)$ time for the amount of data, and $\bigo(m^2)$ memory for the length of the regular expression.
	\end{remark}

	\begin{remark}
		Mention the use of regular language parsing in the design of compilers and interpreters, and mention the ubiquity of \texttt{grep}.
	\end{remark}
% section motivation (end)
\meo{
	Look into any background or motivation that might be good to include.
	Anything about the importance of Kleene's theorem to the implementation of actual software might be good to include.
	But I worry about including too much practical CS in what really should be a math paper.

	It might be cool to cover the minimization stuff if I have extra time and space---although it would add a lot of content to the paper.

}



\end{document}
