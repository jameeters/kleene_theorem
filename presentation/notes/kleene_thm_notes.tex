\documentclass[12pt, twoside, letterpaper]{article}
\usepackage[margin = 1 in]{geometry}

% Lets me copy-paste from beamer without having to find and replace.
\newcommand{\note}[2][a]{\begin{itemize}#2\end{itemize}}

\begin{document}

We start off with the basic definitions of formal language theory.
\begin{itemize}
	\item Symbol
	\begin{itemize}
		\item The most basic unit of formal language theory
		\item like letters or numerals
		\item Nothing you can break them down into and nothing you can combine to make them.
	\end{itemize}

	\item Alphabet
	\begin{itemize}
		\item A set of symbols
		\item the Latin alphabet
		\item 0 and 1 for binary numbers
		\item Alphabets are nonempty and finite
		\item When we need a symbol, we use sigma to represent an alphabet
	\end{itemize}
\end{itemize}

We use symbols from an alphabet to form strings

\begin{itemize}
	\item Strings
	\begin{itemize}
		\item Aka words
		\item sequences of symbols
		\item ``over an alphabet'' means made of symbols from that alphabet
		\item We assume finite length, but not nonempty:
		\item the empty string 
		\begin{itemize}
			\item the string of length 0: notation on the slide.
			\item contains no symbols
			\item considered a string over any alphabet
		\end{itemize}
	\end{itemize}
\end{itemize}
\begin{itemize}
	\item Languages
	\begin{itemize}
		\item sets of strings
		\item Any set of strings over an alphabet is a language over that alphabet.
		\item The empty set is the language that contains no strings, considered a language over any alphabet.
	\end{itemize}
	\item A language could be any set of strings --- you could specify all its members
	\item We are interested in languages formed according to some rules, so that you can describe the language without listing all its members.

\clearpage

\end{itemize}

\begin{itemize}
	\item The regular languages are formed using three set operations
	\item The first is union --- everyone should be familiar.
	\item Concatenation
	\begin{itemize}
		\item Strings
		\begin{itemize}
			\item just the second string bolted onto the end of the first
			\item you can concatenate ``light'' and ``bulb'' to get lightbulb \\ or ``bulb'' and ``light'' to get ``bulblight''
		\end{itemize}
		\item Languages
		\begin{itemize}
			\item The concatenation of two languages is their set product
			\item The set of all strings formed by concatenating some word of the first language with some word of the second language.
		\end{itemize}
	\end{itemize}
	\item Before we move on, I want to go over a few properties of concatenation, for both words and languages.
	\item Empty string
	\begin{itemize}
		\item The concatenation of any word with the empty string is itself.
		\item the product of a language and the set of just the empty string is the original language
	\end{itemize}

	\item Associative
	\begin{itemize}
		\item Concatenation is an associative operation on strings, which means that it is associative on languages as well.
	\end{itemize}

	\item Exponential notation
	\begin{itemize}
		\item We use exponential notation to represent a string or language concatenated with itself
		\item A string or language to the zero is the empty string
	\end{itemize}
\end{itemize}

With that notation out of the way, the last operation should make more sense.
\begin{itemize}
	\item Kleene closure
	\begin{itemize}
		\item 
	\end{itemize}
\end{itemize}

\end{document}
